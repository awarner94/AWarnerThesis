%%%%%%%%%%%%%%%%%%%%%%%%%%%%%%%%%%%%%%%%%%%%%%%%%%%%%%%%%%%%%%%%%%%%%%
% Template for a UBC-compliant dissertation
% At the minimum, you will need to change the information found
% after the "Document meta-data"
%
%!TEX TS-program = pdflatex
%!TEX encoding = UTF-8 Unicode

%% The ubcdiss class provides several options:
%%   gpscopy (aka fogscopy)
%%       set parameters to exactly how GPS specifies
%%         * single-sided
%%         * page-numbering starts from title page
%%         * the lists of figures and tables have each entry prefixed
%%           with 'Figure' or 'Table'
%%       This can be tested by `\ifgpscopy ... \else ... \fi'
%%   10pt, 11pt, 12pt
%%       set default font size
%%   oneside, twoside
%%       whether to format for single-sided or double-sided printing
%%   balanced
%%       when double-sided, ensure page content is centred
%%       rather than slightly offset (the default)
%%   singlespacing, onehalfspacing, doublespacing
%%       set default inter-line text spacing; the ubcdiss class
%%       provides \textspacing to revert to this configured spacing\emph{•}
%%   draft
%%       disable more intensive processing, such as including
%%       graphics, etc.
%%

% For submission to GPS
\documentclass[gpscopy,onehalfspacing,11pt]{ubcdiss}

% For your own copies (looks nicer)
% \documentclass[balanced,twoside,11pt]{ubcdiss}

%%%%%%%%%%%%%%%%%%%%%%%%%%%%%%%%%%%%%%%%%%%%%%%%%%%%%%%%%%%%%%%%%%%%%%
%%%%%%%%%%%%%%%%%%%%%%%%%%%%%%%%%%%%%%%%%%%%%%%%%%%%%%%%%%%%%%%%%%%%%%
%%
%% FONTS:
%% 
%% The defaults below configures Times Roman for the serif font,
%% Helvetica for the sans serif font, and Courier for the
%% typewriter-style font.  Configuring fonts can be time
%% consuming; we recommend skipping to END FONTS!
%% 
%% If you're feeling brave, have lots of time, and wish to use one
%% your platform's native fonts, see the commented out bits below for
%% XeTeX/XeLaTeX.  This is not for the faint at heart. 
%% (And shouldn't you be writing? :-)
%%

%% NFSS font specification (New Font Selection Scheme)
\usepackage{times,mathptmx,courier}
\usepackage[scaled=.92]{helvet}

%% Math or theory people may want to include the handy AMS macros
\usepackage{amssymb}
\usepackage{amsmath}
\usepackage{amsfonts}

%% My new math commands
\DeclareMathOperator{\arcsinh}{arcsinh}
\DeclareMathOperator{\arctanh}{arctanh}

% \usepackage{bm}
\usepackage{xparse}
\usepackage{steinmetz}

% For word count
\usepackage{bashful}

%% BibLatex
%\usepackage[sorting=none,backend=bibtex,style=numeric]{biblatex}
\usepackage[sorting=none,backend=bibtex,style=phys,biblabel=brackets]{biblatex}
\addbibresource{biblio.bib}

%$ Command for complex i:
\newcommand{\iu}{{i\mkern1mu}}

%% The pifont package provides access to the elements in the dingbat font.   
%% Use \ding{##} for a particular dingbat (see p7 of psnfss2e.pdf)
%%   Useful:
%%     51,52 different forms of a checkmark
%%     54,55,56 different forms of a cross (saltyre)
%%     172-181 are 1-10 in open circle (serif)
%%     182-191 are 1-10 black circle (serif)
%%     192-201 are 1-10 in open circle (sans serif)
%%     202-211 are 1-10 in black circle (sans serif)
%% \begin{dinglist}{##}\item... or dingautolist (which auto-increments)
%% to create a bullet list with the provided character.
\usepackage{pifont}
\usepackage{array}

%%%%%%%%%%%%%%%%%%%%%%%%%%%%%%%%%%%%%%%%%%%%%%%%%%%%%%%%%%%%%%%%%%%%%%
%% Configure fonts for XeTeX / XeLaTeX using the fontspec package.
%% Be sure to check out the fontspec documentation.
%\usepackage{fontspec,xltxtra,xunicode}	% required
%\defaultfontfeatures{Mapping=tex-text}	% recommended
%% Minion Pro and Myriad Pro are shipped with some versions of
%% Adobe Reader.  Adobe representatives have commented that these
%% fonts can be used outside of Adobe Reader.
%\setromanfont[Numbers=OldStyle]{Minion Pro}
%\setsansfont[Numbers=OldStyle,Scale=MatchLowercase]{Myriad Pro}
%\setmonofont[Scale=MatchLowercase]{Andale Mono}

%% Other alternatives:
%\setromanfont[Mapping=tex-text]{Adobe Caslon}
%\setsansfont[Scale=MatchLowercase]{Gill Sans}
%\setsansfont[Scale=MatchLowercase,Mapping=tex-text]{Futura}
%\setmonofont[Scale=MatchLowercase]{Andale Mono}
%\newfontfamily{\SYM}[Scale=0.9]{Zapf Dingbats}
%% END FONTS
%%%%%%%%%%%%%%%%%%%%%%%%%%%%%%%%%%%%%%%%%%%%%%%%%%%%%%%%%%%%%%%%%%%%%%
%%%%%%%%%%%%%%%%%%%%%%%%%%%%%%%%%%%%%%%%%%%%%%%%%%%%%%%%%%%%%%%%%%%%%%



%%%%%%%%%%%%%%%%%%%%%%%%%%%%%%%%%%%%%%%%%%%%%%%%%%%%%%%%%%%%%%%%%%%%%%
%%%%%%%%%%%%%%%%%%%%%%%%%%%%%%%%%%%%%%%%%%%%%%%%%%%%%%%%%%%%%%%%%%%%%%
%%
%% Recommended packages
%%
\usepackage{checkend}	% better error messages on left-open environments
\usepackage{graphicx}	% for incorporating external images
\graphicspath{ {./images/} }

%% booktabs: provides some special commands for typesetting tables as used
%% in excellent journals.  Ignore the examples in the Lamport book!
\usepackage{booktabs}

%% listings: useful support for including source code listings, with
%% optional special keyword formatting.  The \lstset{} causes
%% the text to be typeset in a smaller sans serif font, with
%% proportional spacing.
\usepackage{listings}
\lstset{basicstyle=\sffamily\scriptsize,showstringspaces=false,fontadjust}

%% The acronym package provides support for defining acronyms, providing
%% their expansion when first used, and building glossaries.  See the
%% example in glossary.tex and the example usage throughout the example
%% document.
%% NOTE: to use \MakeTextLowercase in the \acsfont command below,
%%   we *must* use the `nohyperlinks' option -- it causes errors with
%%   hyperref otherwise.  See Section 5.2 in the ``LaTeX 2e for Class
%%   and Package Writers Guide'' (clsguide.pdf) for details.
\usepackage[printonlyused,nohyperlinks]{acronym}
%% The ubcdiss.cls loads the `textcase' package which provides commands
%% for upper-casing and lower-casing text.  The following causes
%% the acronym package to typeset acronyms in small-caps
%% as recommended by Bringhurst.
\renewcommand{\acsfont}[1]{{\scshape \MakeTextLowercase{#1}}}

%% color: add support for expressing colour models.  Grey can be used
%% to great effect to emphasize other parts of a graphic or text.
%% For an excellent set of examples, see Tufte's "Visual Display of
%% Quantitative Information" or "Envisioning Information".
\usepackage{color}
\definecolor{greytext}{gray}{0.5}

%% comment: provides a new {comment} environment: all text inside the
%% environment is ignored.
%%   \begin{comment} ignored text ... \end{comment}
\usepackage{comment}

%% The natbib package provides more sophisticated citing commands
%% such as \citeauthor{} to provide the author names of a work,
%% \citet{} to produce an author-and-reference citation,
%% \citep{} to produce a parenthetical citation.
%% We use \citeeg{} to provide examples
%\usepackage[numbers]{natbib}
\newcommand{\citeeg}[1]{\citep[e.g.,][]{#1}}

%% The titlesec package provides commands to vary how chapter and
%% section titles are typeset.  The following uses more compact
%% spacings above and below the title.  The titleformat that follow
%% ensure chapter/section titles are set in singlespace.
\usepackage[compact]{titlesec}
\titleformat*{\section}{\singlespacing\raggedright\bfseries\Large}
\titleformat*{\subsection}{\singlespacing\raggedright\bfseries\large}
\titleformat*{\subsubsection}{\singlespacing\raggedright\bfseries}
\titleformat*{\paragraph}{\singlespacing\raggedright\itshape}

%% The caption package provides support for varying how table and
%% figure captions are typeset.
\usepackage[format=hang,indention=-1cm,labelfont={bf},margin=1em]{caption}

%% url: for typesetting URLs and smart(er) hyphenation.
%% \url{http://...} 
\usepackage{url}
\urlstyle{sf}	% typeset urls in sans-serif


%%%%%%%%%%%%%%%%%%%%%%%%%%%%%%%%%%%%%%%%%%%%%%%%%%%%%%%%%%%%%%%%%%%%%%
%%%%%%%%%%%%%%%%%%%%%%%%%%%%%%%%%%%%%%%%%%%%%%%%%%%%%%%%%%%%%%%%%%%%%%
%%
%% Possibly useful packages: you may need to explicitly install
%% these from CTAN if they aren't part of your distribution;
%% teTeX seems to ship with a smaller base than MikTeX and MacTeX.
%%
%\usepackage{pdfpages}	% insert pages from other PDF files
%\usepackage{longtable}	% provide tables spanning multiple pages
%\usepackage{chngpage}	% support changing the page widths on demand
%\usepackage{tabularx}	% an enhanced tabular environment

%% enumitem: support pausing and resuming enumerate environments.
%\usepackage{enumitem}

%% rotating: provides two environments, sidewaystable and sidewaysfigure,
%% for typesetting tables and figures in landscape mode.  
%\usepackage{rotating}

%% subfig: provides for including subfigures within a figure,
%% and includes being able to separately reference the subfigures.
%\usepackage{subfig}

%% ragged2e: provides several new new commands \Centering, \RaggedLeft,
%% \RaggedRight and \justifying and new environments Center, FlushLeft,
%% FlushRight and justify, which set ragged text and are easily
%% configurable to allow hyphenation.
%\usepackage{ragged2e}

%% The ulem package provides a \sout{} for striking out text and
%% \xout for crossing out text.  The normalem and normalbf are
%% necessary as the package messes with the emphasis and bold fonts
%% otherwise.
%\usepackage[normalem,normalbf]{ulem}    % for \sout

%%%%%%%%%%%%%%%%%%%%%%%%%%%%%%%%%%%%%%%%%%%%%%%%%%%%%%%%%%%%%%%%%%%%%%
%% HYPERREF:
%% The hyperref package provides for embedding hyperlinks into your
%% document.  By default the table of contents, references, citations,
%% and footnotes are hyperlinked.
%%
%% Hyperref provides a very handy command for doing cross-references:
%% \autoref{}.  This is similar to \ref{} and \pageref{} except that
%% it automagically puts in the *type* of reference.  For example,
%% referencing a figure's label will put the text `Figure 3.4'.
%% And the text will be hyperlinked to the appropriate place in the
%% document.
%%
%% Generally hyperref should appear after most other packages

%% The following puts hyperlinks in very faint grey boxes.
%% The `pagebackref' causes the references in the bibliography to have
%% back-references to the citing page; `backref' puts the citing section
%% number.  See further below for other examples of using hyperref.
%% 2009/12/09: now use `linktocpage' (Jacek Kisynski): GPS now prefers
%%   that the ToC, LoF, LoT place the hyperlink on the page number,
%%   rather than the entry text.
\usepackage[bookmarks,bookmarksnumbered,%
    allbordercolors={0.8 0.8 0.8},%
    linktocpage%
    ]{hyperref}
%% The following change how the the back-references text is typeset in a
%% bibliography when `backref' or `pagebackref' are used
%%
%% Change \nocitations if you'd like some text shown where there
%% are no citations found (e.g., pulled in with \nocite{xxx})
\newcommand{\nocitations}{\relax}
%%\newcommand{\nocitations}{No citations}
%%
%\renewcommand*{\backref}[1]{}% necessary for backref < 1.33
%\renewcommand*{\backrefsep}{,~}%
%\renewcommand*{\backreftwosep}{,~}% ', and~'
%\renewcommand*{\backreflastsep}{,~}% ' and~'
%\renewcommand*{\backrefalt}[4]{%
%\textcolor{greytext}{\ifcase #1%
%\nocitations%
%\or
%\(\rightarrow\) page #2%
%\else
%\(\rightarrow\) pages #2%
%\fi}}


%% The following uses most defaults, which causes hyperlinks to be
%% surrounded by colourful boxes; the colours are only visible in
%% PDFs and don't show up when printed:
%\usepackage[bookmarks,bookmarksnumbered]{hyperref}

%% The following disables the colourful boxes around hyperlinks.
%\usepackage[bookmarks,bookmarksnumbered,pdfborder={0 0 0}]{hyperref}

%% The following disables all hyperlinking, but still enabled use of
%% \autoref{}
%\usepackage[draft]{hyperref}

%% The following commands causes chapter and section references to
%% uppercase the part name.
\renewcommand{\chapterautorefname}{Chapter}
\renewcommand{\sectionautorefname}{Section}
\renewcommand{\subsectionautorefname}{Section}
\renewcommand{\subsubsectionautorefname}{Section}
\renewcommand{\equationautorefname}{Equation}

%% If you have long page numbers (e.g., roman numbers in the 
%% preliminary pages for page 28 = xxviii), you might need to
%% uncomment the following and tweak the \@pnumwidth length
%% (default: 1.55em).  See the tocloft documentation at
%% http://www.ctan.org/tex-archive/macros/latex/contrib/tocloft/
% \makeatletter
% \renewcommand{\@pnumwidth}{3em}
% \makeatother

%e Eval at command
\NewDocumentCommand{\evalat}{sO{\big}mm}{%
	\IfBooleanTF{#1}
	{\mleft. #3 \mright|_{#4}}
	{#3#2|_{#4}}%
}

%%%%%%%%%%%%%%%%%%%%%%%%%%%%%%%%%%%%%%%%%%%%%%%%%%%%%%%%%%%%%%%%%%%%%%
%%%%%%%%%%%%%%%%%%%%%%%%%%%%%%%%%%%%%%%%%%%%%%%%%%%%%%%%%%%%%%%%%%%%%%
%%
%% Some special settings that controls how text is typeset
%%
% \raggedbottom		% pages don't have to line up nicely on the last line
% \sloppy		% be a bit more relaxed in inter-word spacing
% \clubpenalty=10000	% try harder to avoid orphans
% \widowpenalty=10000	% try harder to avoid widows
% \tolerance=1000

%% And include some of our own useful macros
% This file provides examples of some useful macros for typesetting
% dissertations.  None of the macros defined here are necessary beyond
% for the template documentation, so feel free to change, remove, and add
% your own definitions.
%
% We recommend that you define macros to separate the semantics
% of the things you write from how they are presented.  For example,
% you'll see definitions below for a macro \file{}: by using
% \file{} consistently in the text, we can change how filenames
% are typeset simply by changing the definition of \file{} in
% this file.
% 
%% The following is a directive for TeXShop to indicate the main file
%%!TEX root = diss.tex

\newcommand{\NA}{\textsc{n/a}}	% for "not applicable"
\newcommand{\eg}{e.g.,\ }	% proper form of examples (\eg a, b, c)
\newcommand{\ie}{i.e.,\ }	% proper form for that is (\ie a, b, c)
\newcommand{\etal}{\emph{et al}}

% Some useful macros for typesetting terms.
\newcommand{\file}[1]{\texttt{#1}}
\newcommand{\class}[1]{\texttt{#1}}
\newcommand{\latexpackage}[1]{\href{http://www.ctan.org/macros/latex/contrib/#1}{\texttt{#1}}}
\newcommand{\latexmiscpackage}[1]{\href{http://www.ctan.org/macros/latex/contrib/misc/#1.sty}{\texttt{#1}}}
\newcommand{\env}[1]{\texttt{#1}}
\newcommand{\BibTeX}{Bib\TeX}

% Define a command \doi{} to typeset a digital object identifier (DOI).
% Note: if the following definition raise an error, then you likely
% have an ancient version of url.sty.  Either find a more recent version
% (3.1 or later work fine) and simply copy it into this directory,  or
% comment out the following two lines and uncomment the third.
\DeclareUrlCommand\DOI{}
\newcommand{\doi}[1]{\href{http://dx.doi.org/#1}{\DOI{doi:#1}}}
%\newcommand{\doi}[1]{\href{http://dx.doi.org/#1}{doi:#1}}

% Useful macro to reference an online document with a hyperlink
% as well with the URL explicitly listed in a footnote
% #1: the URL
% #2: the anchoring text
\newcommand{\webref}[2]{\href{#1}{#2}\footnote{\url{#1}}}

% epigraph is a nice environment for typesetting quotations
\makeatletter
\newenvironment{epigraph}{%
	\begin{flushright}
	\begin{minipage}{\columnwidth-0.75in}
	\begin{flushright}
	\@ifundefined{singlespacing}{}{\singlespacing}%
    }{
	\end{flushright}
	\end{minipage}
	\end{flushright}}
\makeatother

% \FIXME{} is a useful macro for noting things needing to be changed.
% The following definition will also output a warning to the console
\newcommand{\FIXME}[1]{\typeout{**FIXME** #1}\textbf{[FIXME: #1]}}

% END


%%%%%%%%%%%%%%%%%%%%%%%%%%%%%%%%%%%%%%%%%%%%%%%%%%%%%%%%%%%%%%%%%%%%%%
%%%%%%%%%%%%%%%%%%%%%%%%%%%%%%%%%%%%%%%%%%%%%%%%%%%%%%%%%%%%%%%%%%%%%%
%%
%% Document meta-data: be sure to also change the \hypersetup information
%%

\title{Defect and Terrace Characterization of $PtSn_4$ using Scanning Tunneling Microscopy and Spectroscopy}
%\subtitle{If you want a subtitle}

\author{Ashley Nicole Warner}
\previousdegree{B.Sc., UC San Diego, 2018}


% What is this dissertation for?
\degreetitle{Masters of Science}

\institution{The University of British Columbia}
\campus{Vancouver}

\faculty{The Faculty of Graduate and Postdoctoral Studies}
\department{Physics}
\submissionmonth{April}
\submissionyear{2022}

% details of your examining committee
\examiningcommittee{Douglas Bonn, Physics and Astronomy, UBC}{Supervisor}

% details of your supervisory committee
\supervisorycommittee{Sarah Burke, Physics and Astronomy, UBC}%
    {Supervisory Committee Member}


%% hyperref package provides support for embedding meta-data in .PDF
%% files
\hypersetup{
  pdftitle={PtSn4defectsOnSTM_AshleyNicoleWarner (DRAFT: \today)},
  pdfauthor={Ashley Nicole Warner},
  pdfkeywords={PtSn4, stm, spm, scanning tunneling microscopy, quantum materials, topological insulator, topological semimetal}
}

%%%%%%%%%%%%%%%%%%%%%%%%%%%%%%%%%%%%%%%%%%%%%%%%%%%%%%%%%%%%%%%%%%%%%%
%%%%%%%%%%%%%%%%%%%%%%%%%%%%%%%%%%%%%%%%%%%%%%%%%%%%%%%%%%%%%%%%%%%%%%
%% 
%% The document content
%%

%% LaTeX's \includeonly commands causes any uses of \include{} to only
%% include files that are in the list.  This is helpful to produce
%% subsets of your thesis (e.g., for committee members who want to see
%% the dissertation chapter by chapter).  It also saves time by 
%% avoiding reprocessing the entire file.
%\includeonly{intro,conclusions}
%\includeonly{discussion}

\begin{document}

%%%%%%%%%%%%%%%%%%%%%%%%%%%%%%%%%%%%%%%%%%%%%%%%%%
%% From Thesis Components: Tradtional Thesis
%% <http://www.grad.ubc.ca/current-students/dissertation-thesis-preparation/order-components>

% Preliminary Pages (numbered in lower case Roman numerals)
%    1. Title page (mandatory)
\maketitle

%    2. Committee page (mandatory): lists supervisory committee and,
%    if applicable, the examining committee
\makecommitteepage

%    3. Abstract (mandatory - maximum 350 words)
%% The following is a directive for TeXShop to indicate the main file
%%!TEX root = diss.tex

\chapter{Abstract}

This document provides brief instructions for using the \class{ubcdiss}
class to write a \acs{UBC}-conformant dissertation in \LaTeX.  This
document is itself written using the \class{ubcdiss} class and is
intended to serve as an example of writing a dissertation in \LaTeX.
This document has embedded \acp{URL} and is intended to be viewed
using a computer-based \ac{PDF} reader.

Note: Abstracts should generally try to avoid using acronyms.

Note: at \ac{UBC}, both the \ac{GPS} Ph.D. defence programme and the
Library's online submission system restricts abstracts to 350
words.

\ifgpscopy
  This document was typeset in \texttt{gpscopy} mode.
\else
  This document was typeset in non-\texttt{gpscopy} mode.
\fi

% Consider placing version information if you circulate multiple drafts
%\vfill
%\begin{center}
%\begin{sf}
%\fbox{Revision: \today}
%\end{sf}
%\end{center}

\cleardoublepage

%    4. Lay Summary (Effective May 2017, mandatory - maximum 150 words)
%% The following is a directive for TeXShop to indicate the main file
%%!TEX root = diss.tex

%% https://www.grad.ubc.ca/current-students/dissertation-thesis-preparation/preliminary-pages
%% 
%% LAY SUMMARY Effective May 2017, all theses and dissertations must
%% include a lay summary.  The lay or public summary explains the key
%% goals and contributions of the research/scholarly work in terms that
%% can be understood by the general public. It must not exceed 150
%% words in length.

\chapter{Lay Summary}

The lay or public summary explains the key goals and contributions of
the research\slash{}scholarly work in terms that can be understood by the
general public. It must not exceed 150 words in length.

\cleardoublepage

%    5. Preface
%% The following is a directive for TeXShop to indicate the main file
%%!TEX root = diss.tex

\chapter{Preface}

At \ac{UBC}, a preface may be required.  Be sure to check the
\ac{GPS} guidelines as they may have specific content to be included.

\cleardoublepage

%    6. Table of contents (mandatory - list all items in the preliminary pages
%    starting with the abstract, followed by chapter headings and
%    subheadings, bibliographies and appendices)
\tableofcontents
\cleardoublepage	% required by tocloft package

%    7. List of tables (mandatory if thesis has tables)
\listoftables
\cleardoublepage	% required by tocloft package

%    8. List of figures (mandatory if thesis has figures)
\listoffigures
\cleardoublepage	% required by tocloft package

%    9. List of illustrations (mandatory if thesis has illustrations)
%   10. Lists of symbols, abbreviations or other (optional)

%   11. Glossary (optional)
%% The following is a directive for TeXShop to indicate the main file
%%!TEX root = diss.tex

\chapter{Glossary}

This glossary uses the handy \latexpackage{acroynym} package to automatically
maintain the glossary.  It uses the package's \texttt{printonlyused}
option to include only those acronyms explicitly referenced in the
\LaTeX\ source.  To change how the acronyms are rendered, change the
\verb+\acsfont+ definition in \verb+diss.tex+.

% use \acrodef to define an acronym, but no listing
\acrodef{UI}{user interface}
\acrodef{UBC}{University of British Columbia}

% The acronym environment will typeset only those acronyms that were
% *actually used* in the course of the document
\begin{acronym}[ANOVA]

\acro{DOI}{Document Object Identifier\acroextra{ (see
    \url{http://doi.org})}}
\acro{GPS}[GPS]{Graduate and Postdoctoral Studies}
\acro{PDF}[PDF]{Portable Document Format}
\acro{STM}[STM]{Scanning Tunneling Microscopy}
\acro{STS}[SPM]{Scanning Probing Microscopy}
\end{acronym}

% You can also use \newacro{}{} to only define acronyms
% but without explictly creating a glossary
% 
% \newacro{ANOVA}[ANOVA]{Analysis of Variance\acroextra{, a set of
%   statistical techniques to identify sources of variability between groups.}}
% \newacro{API}[API]{application programming interface}
% \newacro{GOMS}[GOMS]{Goals, Operators, Methods, and Selection\acroextra{,
%   a framework for usability analysis.}}
% \newacro{TLX}[TLX]{Task Load Index\acroextra{, an instrument for gauging
%   the subjective mental workload experienced by a human in performing
%   a task.}}
% \newacro{UI}[UI]{user interface}
% \newacro{UML}[UML]{Unified Modelling Language}
% \newacro{W3C}[W3C]{World Wide Web Consortium}
% \newacro{XML}[XML]{Extensible Markup Language}
	% always input, since other macros may rely on it

\textspacing		% begin one-half or double spacing

%   12. Acknowledgements
%% The following is a directive for TeXShop to indicate the main file
%%!TEX root = diss.tex

\chapter{Acknowledgments}

Thank those people who helped you. 

Don't forget your parents or loved ones.

You may wish to acknowledge your funding sources.


%   13. Dedication
\clearpage 
\begin{center}
    \thispagestyle{empty}
    \vspace*{\fill}
    % \emph{This one's for me.}
    \vspace*{\fill}
\end{center}
\clearpage
% Body of Thesis (not all sections may apply)
\mainmatter

\acresetall	% reset all acronyms used so far

%    1. Introduction
% The following is a directive for TeXShop to indicate the main file
% !TEX root = diss.tex

% \begin{center}
%     % \thispagestyle{empty}
%     \chapter{Introduction to $PtSn_4$ and Topological Semimetals}
%     \label{ch:Introduction}
% \end{center}
% \clearpage
\chapter{Introduction}
\label{ch:Introduction}


In year (forever ago) Bell Labs put silicon and germanium (I think) together and created the first transistor, revolutionizing how the world functioned day to day. Since then material scientists of the like have dove head first into hundreds of different phenomena that arises in condensed matter physics. From superconductors, skyrmions, and topological insulators, the quantum age was born.

Topological materials are one small class in a sea of fascination within condensed matter physics, but have the potential to unlike (reference a bunch of cool shit).

Here we're gonna talk about what topological insulators are, where $PtSn_4$ started, why we care, and what makes it special. Need to find lots of references. Note to future students, download and save the bibs of every paper you ever download ever. Also can I just, never put two figures right next to each other but different files? Like, fuckin lame I don't wanna merge them outside of \LaTeX\ but I guess Imma have too UGH.

%%%%%%%%%%%%%%%%%%%%%%%%%%%%%%%%%%%%%%%%%%%%%%%%%%%%%%%%%%%%%%%%%%%%%%%%%%%%%
\section{Topological Insulators}

\subsection{The Past}
Lorem ipsum dolor sit amet, consectetur adipiscing elit. Vestibulum porta eget augue vitae consectetur. Aenean varius sem id scelerisque convallis. Duis aliquam, neque ut consequat gravida, lacus enim imperdiet dui, nec ultrices turpis tellus commodo erat. Mauris bibendum efficitur posuere. Sed porta nunc vitae justo hendrerit hendrerit. Fusce sodales dolor nec accumsan consectetur. Etiam eu tortor tempus, commodo nibh eget, suscipit magna. Nulla urna urna, condimentum non dignissim nec, lacinia in dui. Nunc eu ipsum rhoncus, sodales neque et, congue leo. Nulla facilisi. Phasellus nec magna vel tortor gravida gravida. Nullam sem eros, dapibus nec ligula eu, consectetur volutpat diam. Phasellus interdum neque sit amet porttitor sagittis. Nam commodo, nibh lobortis molestie blandit, leo elit facilisis augue, in viverra orci nibh et nibh. Nunc fringilla sem leo, id fermentum mi porttitor eget. Phasellus suscipit ultrices leo id fringilla.

    \begin{figure}
        \centering
        \includegraphics[width = 4in, keepaspectratio]{Figures/topoIns.png}
        \caption{Topological stuff. }
        \label{fig:topoIns}
    \end{figure}

Quisque luctus, nulla ut lobortis varius, sapien nulla tempor risus, euismod hendrerit lacus magna nec elit. Suspendisse semper nisl vel augue cursus lobortis. Nunc nec tincidunt urna, ut bibendum arcu. Maecenas sodales eros vel quam lobortis bibendum. Nam efficitur tortor non odio imperdiet congue. Mauris libero quam, lobortis sed ullamcorper id, cursus nec libero. Vivamus efficitur sem eu varius ultrices. Duis diam nulla, porta vel cursus vel, facilisis nec urna. Sed condimentum eleifend nisi eu luctus. Morbi sit amet mi ex. Nullam vulputate porttitor nunc, ac euismod nisl vestibulum at. Ut ornare nisi sit amet turpis accumsan mollis.

\subsection{The Present}

Curabitur rhoncus ullamcorper purus, at euismod sapien commodo sed. Sed pulvinar, justo sit amet laoreet pretium, felis sem vestibulum sem, vitae finibus tellus mi in ex. Praesent ut posuere magna, id consectetur urna. Ut pretium non purus et luctus. Integer sed lacus vitae augue porta venenatis vel a est. Morbi in posuere ex. Pellentesque euismod pharetra augue nec consequat. Curabitur non libero vehicula, condimentum eros eu, sodales massa. Vestibulum mollis risus ac urna convallis, sed porttitor tellus dictum. Nulla turpis est, suscipit ut varius in, porta id enim. Nam mollis tincidunt dictum. Ut nibh mi, fermentum eu laoreet et, commodo in nibh. Vestibulum ante ipsum primis in faucibus orci luctus et ultrices posuere cubilia curae; Ut convallis ex id purus vulputate, id elementum elit aliquam. Ut vel risus varius, molestie est eu, eleifend leo. Integer vitae dictum urna.

Fusce sit amet augue est. Nulla mattis ullamcorper elit. Ut non tellus ac sapien cursus suscipit pharetra ut ante. Integer congue libero nec neque vulputate, eu finibus magna maximus. Aenean elit mi, pellentesque non urna id, mattis feugiat lectus. Fusce at augue mi. Nulla vehicula dignissim dolor, vel consequat massa mattis vel.

Duis in lorem libero. Interdum et malesuada fames ac ante ipsum primis in faucibus. Duis nec urna lacus. Nunc vel augue ac ex tristique volutpat. Nulla bibendum nunc in fermentum condimentum. Nam eu pulvinar nibh. Orci varius natoque penatibus et magnis dis parturient montes, nascetur ridiculus mus. Ut egestas lobortis leo a hendrerit. Etiam elementum dolor in nunc interdum commodo.

Lorem ipsum dolor sit amet, consectetur adipiscing elit. Nam pulvinar rhoncus libero. Orci varius natoque penatibus et magnis dis parturient montes, nascetur ridiculus mus. Pellentesque congue, risus eu viverra scelerisque, nibh dolor porttitor tellus, eu vestibulum ligula lorem vel sem. Proin nisl nibh, blandit quis lorem eget, ultricies tempus lectus. Aliquam erat volutpat. Aenean luctus efficitur arcu id interdum. Mauris iaculis tempus ex in interdum. Phasellus a elementum tortor. Morbi facilisis urna nec ligula sagittis faucibus.
    
    \begin{figure}
        \centering
        \includegraphics[width = 4in, keepaspectratio]{Figures/topoIns2.png}
        \caption{tOpOlOgIcAl InSuLaToR. }
        \label{fig:topoIns2}
    \end{figure}
    
Donec facilisis, magna non fermentum auctor, tellus lacus aliquet tortor, eget pulvinar massa orci lobortis arcu. Proin finibus turpis ultricies ultrices posuere. Lorem ipsum dolor sit amet, consectetur adipiscing elit. Vestibulum tristique posuere neque. Duis scelerisque libero sit amet leo sodales, malesuada ullamcorper tellus ultricies. Cras non dapibus nisi. Integer ante augue, elementum vel interdum non, vestibulum eget magna. Vivamus laoreet, nisl eu iaculis venenatis, urna quam lobortis risus, vel ultrices velit lacus nec mauris. Curabitur volutpat ex in varius interdum. Aenean ut dignissim augue.

\subsection{The Future}

Nullam non tincidunt nulla, in egestas eros. Sed tincidunt, erat eget vehicula finibus, est ante commodo velit, quis mollis ipsum nulla in risus. Vivamus vestibulum lacinia tortor, non consectetur velit convallis eget. In sagittis turpis erat, tempus viverra magna feugiat eget. Phasellus et ante leo. Vestibulum viverra arcu orci, in varius lacus posuere et. Nulla facilisi. Pellentesque volutpat euismod urna sed suscipit. Donec sit amet turpis at sapien tristique egestas. Nullam rhoncus sit amet nisi quis fermentum. Integer non nunc non nisl viverra accumsan. Nullam vel sapien metus. Proin efficitur ultrices sagittis.

    
Fusce lobortis lectus vitae risus laoreet, quis imperdiet est gravida. Fusce et dui sit amet dolor elementum tristique. Etiam felis urna, egestas vel sapien sed, interdum dictum lacus. Nullam eu auctor augue. Aliquam mollis, ante quis mollis rhoncus, elit magna volutpat orci, eget malesuada urna sem sit amet erat. Cras porttitor eros ornare est placerat, in luctus nisi pellentesque. Pellentesque rhoncus augue vel ornare pharetra. Phasellus ac elementum massa.

%%%%%%%%%%%%%%%%%%%%%%%%%%%%%%%%%%%%%%%%%%%%%%%%%%%%%%%%%%%%%%%%%%%%%%%%%%%%%%%%%%
\section{$PtSn_4$}

Nunc eu nulla placerat, posuere arcu ut, elementum tellus. Donec non tortor et metus commodo vulputate rutrum porttitor odio. In sed condimentum purus. Pellentesque pulvinar tortor hendrerit pulvinar euismod. Nam vehicula diam odio, vel molestie ligula dictum et. Cras et orci diam. Etiam quis tortor orci. Donec vel urna purus.

    \begin{figure}
        \centering
        \includegraphics[width = 4.2in, keepaspectratio]{Figures/PtSn4CrystalLattice.jpg}
        \caption{\textbf{(a)} A slanted side profile of a $PtSn_4$ unit cell showing bonds to 			each atoms nearest neighbors. This reveals an obvious plane halfway down the unit cell 			that makes $PtSn_4$ a good probable candidate for cleaving. \textbf{(b)} A vertical 			view of the surface of a $PtSn_4$ unit cell.}
        \label{fig:PtSn_4CrystalLattice}
    \end{figure}

Proin mauris tellus, molestie ut nibh ac, tristique ultrices tortor. Fusce faucibus, justo vel dapibus lobortis, velit justo venenatis augue, sed ornare lectus massa vel nulla. Integer vel tellus arcu. Maecenas congue ac nibh eu eleifend. Pellentesque habitant morbi tristique senectus et netus et malesuada fames ac turpis egestas. Maecenas imperdiet sapien nec lacus sodales faucibus. Ut lacinia est at quam tincidunt, vitae lacinia enim facilisis. Donec auctor turpis nec risus consectetur pharetra. Integer pretium ac nibh id elementum. Aenean suscipit ac tellus eu facilisis. Nunc cursus, lacus id rutrum tristique, ipsum diam maximus tortor, a hendrerit nunc neque eu augue. Orci varius natoque penatibus et magnis dis parturient montes, nascetur ridiculus mus. Suspendisse potenti. Nullam ultricies sed mi vel condimentum. Nam condimentum, lacus quis fringilla suscipit, ligula arcu placerat tortor, a congue ex eros vitae mauris.

    \begin{figure}
        \centering
        \includegraphics[width = 3.5in, keepaspectratio]{Figures/unitCellMeasured.png}
        \caption{A side profile of $PtSn_4$ with common spaces measured and listed.}
        \label{fig:unitCellMeasured}
    \end{figure}
    
    \begin{figure}
        \centering
        \includegraphics[width = 4.2in, keepaspectratio]{Figures/PtSn4Labeling.png}
        \caption{A side profile of $PtSn_4$ denoting the labeling mechanism used and often 						referred to throughout this work.}
        \label{fig:unitCellLabeling}
    \end{figure}

Morbi molestie nunc eleifend pretium volutpat. Donec pharetra nunc ut dignissim ullamcorper. Proin vitae lectus interdum, euismod sapien sed, venenatis nibh. Pellentesque justo urna, gravida vel pulvinar at, bibendum finibus urna. Class aptent taciti sociosqu ad litora torquent per conubia nostra, per inceptos himenaeos. Aenean tincidunt, velit sit amet rutrum semper, mi velit posuere massa, id porta tellus eros non tellus. Nunc blandit dui nisi, a lobortis tortor facilisis condimentum. Aliquam erat volutpat.
=
    \begin{figure}
        \centering
        \includegraphics[width = 3in, keepaspectratio]{Figures/diracNodalLines_PtSn4.png}
        \caption{Simulated fermi surface of bulk vs surface of $PtSn_4$. }
        \label{fig:nodalLines}
    \end{figure}

Vestibulum in mattis leo. Pellentesque nec nibh quis magna consectetur viverra ac et mauris. Nunc porta pretium rutrum. Vivamus hendrerit, lorem sed maximus cursus, velit enim luctus felis, a luctus ante felis non est. Sed dignissim libero orci, nec mollis libero hendrerit vel. Mauris eros lorem, efficitur placerat suscipit quis, tincidunt sed nisl. Proin sollicitudin sem sed eros pretium, in suscipit nulla fermentum. Nulla rutrum et tellus dignissim pretium. Aenean ac sagittis magna. Vestibulum convallis vestibulum diam, imperdiet iaculis dolor faucibus eu. Vivamus mattis turpis in ligula rhoncus gravida. Quisque aliquet, massa nec imperdiet eleifend, arcu felis efficitur nisl, sed malesuada lorem libero sed sem. Vivamus vitae egestas ex, ac condimentum diam. Donec porttitor lacus id metus luctus, sed sagittis augue eleifend. Quisque hendrerit maximus finibus. Fusce mollis sodales porttitor.

Sed rhoncus orci vitae imperdiet dapibus. Ut porta nulla at purus vulputate, non lobortis lectus pretium. Fusce at augue sit amet arcu commodo fringilla. Cras urna odio, volutpat eu feugiat a, maximus tincidunt dui. Nulla semper ligula quis orci ultricies molestie. Suspendisse a nisl auctor, consectetur eros in, facilisis nunc. Praesent pulvinar placerat purus quis pulvinar. Aenean tempus pharetra elit vel viverra. Aenean sagittis, nisi vel pharetra faucibus, mauris ex venenatis quam, vel varius massa nisi sed lacus. Pellentesque habitant morbi tristique senectus et netus et malesuada fames ac turpis egestas. In euismod rhoncus turpis, in varius risus vehicula id.

Ut leo diam, tristique ut luctus et, rutrum vel risus. Praesent vel finibus erat, nec tincidunt elit. Sed auctor tristique turpis, nec lacinia lectus dignissim sed. Pellentesque tristique rutrum consequat. Sed at viverra elit. Vestibulum metus nisl, commodo non bibendum ut, tincidunt in metus. Fusce ligula sapien, iaculis sit amet consequat eget, scelerisque venenatis nulla. Vestibulum ante ipsum primis in faucibus orci luctus et ultrices posuere cubilia curae; Mauris non nisi ac diam varius hendrerit at eleifend lectus. Suspendisse leo nibh, tristique sit amet velit in, elementum molestie lectus. Nullam a turpis molestie eros sodales ultricies non ut risus. Pellentesque habitant morbi tristique senectus et netus et malesuada fames ac turpis egestas. Nunc condimentum ut leo non porttitor. Interdum et malesuada fames ac ante ipsum primis in faucibus.

    \begin{figure}
        \centering
        \includegraphics[width = 4.2in, keepaspectratio]{Figures/bulkVsSurface.png}
        \caption{Simulated fermi surface of bulk vs surface of $PtSn_4$. }
        \label{fig:bulkvssurface}
    \end{figure}

Nunc eu nulla placerat, posuere arcu ut, elementum tellus. Donec non tortor et metus commodo vulputate rutrum porttitor odio. In sed condimentum purus. Pellentesque pulvinar tortor hendrerit pulvinar euismod. Nam vehicula diam odio, vel molestie ligula dictum et. Cras et orci diam. Etiam quis tortor orci. Donec vel urna purus.

Proin mauris tellus, molestie ut nibh ac, tristique ultrices tortor. Fusce faucibus, justo vel dapibus lobortis, velit justo venenatis augue, sed ornare lectus massa vel nulla. Integer vel tellus arcu. Maecenas congue ac nibh eu eleifend. Pellentesque habitant morbi tristique senectus et netus et malesuada fames ac turpis egestas. Maecenas imperdiet sapien nec lacus sodales faucibus. Ut lacinia est at quam tincidunt, vitae lacinia enim facilisis. Donec auctor turpis nec risus consectetur pharetra. Integer pretium ac nibh id elementum. Aenean suscipit ac tellus eu facilisis. Nunc cursus, lacus id rutrum tristique, ipsum diam maximus tortor, a hendrerit nunc neque eu augue. Orci varius natoque penatibus et magnis dis parturient montes, nascetur ridiculus mus. Suspendisse potenti. Nullam ultricies sed mi vel condimentum. Nam condimentum, lacus quis fringilla suscipit, ligula arcu placerat tortor, a congue ex eros vitae mauris.

    \begin{figure}
        \centering
        \includegraphics[width = 3in, keepaspectratio]{Figures/samikshyaData.pdf}
        \caption{data from the growth batch. }
        \label{fig:topoIns}
    \end{figure}

Morbi molestie nunc eleifend pretium volutpat. Donec pharetra nunc ut dignissim ullamcorper. Proin vitae lectus interdum, euismod sapien sed, venenatis nibh. Pellentesque justo urna, gravida vel pulvinar at, bibendum finibus urna. Class aptent taciti sociosqu ad litora torquent per conubia nostra, per inceptos himenaeos. Aenean tincidunt, velit sit amet rutrum semper, mi velit posuere massa, id porta tellus eros non tellus. Nunc blandit dui nisi, a lobortis tortor facilisis condimentum. Aliquam erat volutpat.

    \begin{figure}
        \centering
        \includegraphics[width = 3in, keepaspectratio]{Figures/dft.pdf}
        \caption{DFT, maybe. }
        \label{fig:topoIns}
    \end{figure}

Vestibulum in mattis leo. Pellentesque nec nibh quis magna consectetur viverra ac et mauris. Nunc porta pretium rutrum. Vivamus hendrerit, lorem sed maximus cursus, velit enim luctus felis, a luctus ante felis non est. Sed dignissim libero orci, nec mollis libero hendrerit vel. Mauris eros lorem, efficitur placerat suscipit quis, tincidunt sed nisl. Proin sollicitudin sem sed eros pretium, in suscipit nulla fermentum. Nulla rutrum et tellus dignissim pretium. Aenean ac sagittis magna. Vestibulum convallis vestibulum diam, imperdiet iaculis dolor faucibus eu. Vivamus mattis turpis in ligula rhoncus gravida. Quisque aliquet, massa nec imperdiet eleifend, arcu felis efficitur nisl, sed malesuada lorem libero sed sem. Vivamus vitae egestas ex, ac condimentum diam. Donec porttitor lacus id metus luctus, sed sagittis augue eleifend. Quisque hendrerit maximus finibus. Fusce mollis sodales porttitor.

Sed rhoncus orci vitae imperdiet dapibus. Ut porta nulla at purus vulputate, non lobortis lectus pretium. Fusce at augue sit amet arcu commodo fringilla. Cras urna odio, volutpat eu feugiat a, maximus tincidunt dui. Nulla semper ligula quis orci ultricies molestie. Suspendisse a nisl auctor, consectetur eros in, facilisis nunc. Praesent pulvinar placerat purus quis pulvinar. Aenean tempus pharetra elit vel viverra. Aenean sagittis, nisi vel pharetra faucibus, mauris ex venenatis quam, vel varius massa nisi sed lacus. Pellentesque habitant morbi tristique senectus et netus et malesuada fames ac turpis egestas. In euismod rhoncus turpis, in varius risus vehicula id.

Ut leo diam, tristique ut luctus et, rutrum vel risus. Praesent vel finibus erat, nec tincidunt elit. Sed auctor tristique turpis, nec lacinia lectus dignissim sed. Pellentesque tristique rutrum consequat. Sed at viverra elit. Vestibulum metus nisl, commodo non bibendum ut, tincidunt in metus. Fusce ligula sapien, iaculis sit amet consequat eget, scelerisque venenatis nulla. Vestibulum ante ipsum primis in faucibus orci luctus et ultrices posuere cubilia curae; Mauris non nisi ac diam varius hendrerit at eleifend lectus. Suspendisse leo nibh, tristique sit amet velit in, elementum molestie lectus. Nullam a turpis molestie eros sodales ultricies non ut risus. Pellentesque habitant morbi tristique senectus et netus et malesuada fames ac turpis egestas. Nunc condimentum ut leo non porttitor. Interdum et malesuada fames ac ante ipsum primis in faucibus.

%    2. Main body
\chapter{Methods}
\label{ch:Methods}


\section{The History of the Scanning Tunneling Microscope}
In the late 1970's, physicists Gerd Binnig and Heinrich Rohrer decided to put together the science of quantum tunneling and apply it to a vacuum. \cite{binnig-1981-tunneling} They discovered that not only could this technique acquire spectroscopy data, but also high resolution topographic images of materials. Thus, the first scanning tunneling microscope (STM) was invented and earned them the 1986 Nobel Prize in Physics. \cite{binnig-1982-stm, binnig-1987-stm} Over 40 years later, something that ``should not have worked in principle" now has given the science community boundless insight into interesting materials physics, ranging from atom manipulation to (idk some other crazy shit) \cite{binnig-1987-stm} (cite a bunch of cool STM science articles)


\\
(figure of first STM?)
\\

\section{Scanning Tunneling Microscopy Fundamentals}
The bases of a STM centers around the disparity between classical and quantum mechanics, specifically quantum tunneling. Classical mechanics is much more intuitive; a basketball is thrown against a brick barrier with energy $T_{basketball}$, it will bounce against the wall if the strength of the wall, say $U_{barrier}$ is less than the incoming energy of the basketball, or $T_{basketball} < U_{barrier}$.


Classic vs Quantum. Everything can be described as a wave function. On a macro scale, a basketball can only penetrate if it has enough energy to break through the barrier. If $T_{basketball} > U_{barrier}$, if you will.

    \begin{figure}[!h]
        \centering
        \includegraphics[width = 5in, keepaspectratio]{Figures/classical_Basketball.png}
        \caption{A set of cartoon images providing an analogy of classical mechanics. \textbf{(a)} If the energy of the basketball is less than the energy required to keep the brick wall together (ie, $T_{basketball} > U_{barrier}$, the basketball will not penetrate the brick wall.  \textbf{(b)} If $T_{basketball} < U_{barrier}$, then the energy will go into destroying the brick wall and the basketball will pass through.}
        \label{fig:classicBasket}
    \end{figure}


    \begin{figure}
        \centering
        \includegraphics[width = 5in, keepaspectratio]{Figures/e_Tunnel_Through_Wall_compare.png}
        \caption{\textbf{(top)} Cartoon example of an electron tunneling through a barrier. \textbf{(bottom)} Analogous wave equation of the same transmission phenomenon.}
        \label{fig:eTunneling}
    \end{figure}


\subsection{The Physics Behind It}

Nullam non tincidunt nulla, in egestas eros. Sed tincidunt, erat eget vehicula finibus, est ante commodo velit, quis mollis ipsum nulla in risus. Vivamus vestibulum lacinia tortor, non consectetur velit convallis eget. In sagittis turpis erat, tempus viverra magna feugiat eget. Phasellus et ante leo. Vestibulum viverra arcu orci, in varius lacus posuere et. Nulla facilisi. Pellentesque volutpat euismod urna sed suscipit. Donec sit amet turpis at sapien tristique egestas. Nullam rhoncus sit amet nisi quis fermentum. Integer non nunc non nisl viverra accumsan. Nullam vel sapien metus. Proin efficitur ultrices sagittis.

    \begin{figure}
        \centering
        \includegraphics[width = 4.5in, keepaspectratio]{Figures/stmCircuitEx.png}
        \caption{Circuit tunneling analogy. }
        \label{fig:stm}
    \end{figure}

Donec facilisis, magna non fermentum auctor, tellus lacus aliquet tortor, eget pulvinar massa orci lobortis arcu. Proin finibus turpis ultricies ultrices posuere. Lorem ipsum dolor sit amet, consectetur adipiscing elit. Vestibulum tristique posuere neque. Duis scelerisque libero sit amet leo sodales, malesuada ullamcorper tellus ultricies. Cras non dapibus nisi. Integer ante augue, elementum vel interdum non, vestibulum eget magna. Vivamus laoreet, nisl eu iaculis venenatis, urna quam lobortis risus, vel ultrices velit lacus nec mauris. Curabitur volutpat ex in varius interdum. Aenean ut dignissim augue.

\subsection{The Bare Necessities}

Nullam non tincidunt nulla, in egestas eros. Sed tincidunt, erat eget vehicula finibus, est ante commodo velit, quis mollis ipsum nulla in risus. Vivamus vestibulum lacinia tortor, non consectetur velit convallis eget. In sagittis turpis erat, tempus viverra magna feugiat eget. Phasellus et ante leo. Vestibulum viverra arcu orci, in varius lacus posuere et. Nulla facilisi. Pellentesque volutpat euismod urna sed suscipit. Donec sit amet turpis at sapien tristique egestas. Nullam rhoncus sit amet nisi quis fermentum. Integer non nunc non nisl viverra accumsan. Nullam vel sapien metus. Proin efficitur ultrices sagittis.

Fusce lobortis lectus vitae risus laoreet, quis imperdiet est gravida. Fusce et dui sit amet dolor elementum tristique. Etiam felis urna, egestas vel sapien sed, interdum dictum lacus. Nullam eu auctor augue. Aliquam mollis, ante quis mollis rhoncus, elit magna volutpat orci, eget malesuada urna sem sit amet erat. Cras porttitor eros ornare est placerat, in luctus nisi pellentesque. Pellentesque rhoncus augue vel ornare pharetra. Phasellus ac elementum massa.

	\begin{enumerate}
		\item \textbf{Atomically Sharp Tip} \\ Fusce lobortis lectus vitae risus laoreet, quis imperdiet est gravida. Fusce et dui sit amet dolor elementum tristique. Etiam felis urna, egestas vel sapien sed, interdum dictum lacus. Nullam eu auctor augue. Aliquam mollis, ante quis mollis rhoncus, elit magna volutpat orci, eget malesuada urna sem
		
    		\begin{figure}[!h]
        		\centering
        		\includegraphics[width = 3in, keepaspectratio]{Figures/beforeAndAfterTip.png}
        		\caption{\textbf{(a)} A picture taken June 17th, 2021 of the STM tip through a microscope used to take a majority of the data in this work. \textbf{(b)} The same tip after being used by the STM for six months, taken mid December 2021.}
        		\label{fig:beforeAndAfter}
    		\end{figure}    
    
		\item \textbf{Applied Voltage in DC Circuit} \\ Nunc eu nulla placerat, posuere arcu ut, elementum tellus. Donec non tortor et metus commodo vulputate rutrum porttitor odio. In sed condimentum purus. Pellentesque pulvinar tortor hendrerit pulvinar euismod. Nam vehicula diam odio, vel molestie ligula dictum et. Cras et orci diam. Etiam quis tortor orci. Donec vel urna purus.
		
			\begin{figure}[!h]
        		\centering
        		\includegraphics[width = 4in, keepaspectratio]{Figures/stmTopoZ.png}
        		\caption{feedback keeping z steady.}
        		\label{fig:stm}
    		\end{figure}
		
		\item \textbf{Feedback Look} \\ aldksfja;dksflj
		
    		\begin{figure}[!h]
        		\centering
        		\includegraphics[width = 3in, keepaspectratio]{Figures/stmFeedbackLoop.png}
        		\caption{STM example.}
        		\label{fig:stmCircuit}
    		\end{figure}
		
		\item \textbf{Conducting Sample} \\ Ya gotta close the circuit somehow man.
	\end{enumerate}


Nunc eu nulla placerat, posuere arcu ut, elementum tellus. Donec non tortor et metus commodo vulputate rutrum porttitor odio. In sed condimentum purus. Pellentesque pulvinar tortor hendrerit pulvinar euismod. Nam vehicula diam odio, vel molestie ligula dictum et. Cras et orci diam. Etiam quis tortor orci. Donec vel urna purus.

Proin mauris tellus, molestie ut nibh ac, tristique ultrices tortor. Fusce faucibus, justo vel dapibus lobortis, velit justo venenatis augue, sed ornare lectus massa vel nulla. Integer vel tellus arcu. Maecenas congue ac nibh eu eleifend. Pellentesque habitant morbi tristique senectus et netus et malesuada fames ac turpis egestas. Maecenas imperdiet sapien nec lacus sodales faucibus. Ut lacinia est at quam tincidunt, vitae lacinia enim facilisis. Donec auctor turpis nec risus consectetur pharetra. Integer pretium ac nibh id elementum. Aenean suscipit ac tellus eu facilisis. Nunc cursus, lacus id rutrum tristique, ipsum diam maximus tortor, a hendrerit nunc neque eu augue. Orci varius natoque penatibus et magnis dis parturient montes, nascetur ridiculus mus. Suspendisse potenti. Nullam ultricies sed mi vel condimentum. Nam condimentum, lacus quis fringilla suscipit, ligula arcu placerat tortor, a congue ex eros vitae mauris.

Proin mauris tellus, molestie ut nibh ac, tristique ultrices tortor. Fusce faucibus, justo vel dapibus lobortis, velit justo venenatis augue, sed ornare lectus massa vel nulla. Integer vel tellus arcu. Maecenas congue ac nibh eu eleifend. Pellentesque habitant morbi tristique senectus et netus et malesuada fames ac turpis egestas. Maecenas imperdiet sapien nec lacus sodales faucibus.




%%%%%%%%%%%%%%%%%%%%%%%%%%%%%%%%%%%%%%%%%%%%%%%%%%%%%%%%%%%%%%%%%%%%
\section{How to Topo}

Morbi molestie nunc eleifend pretium volutpat. Donec pharetra nunc ut dignissim ullamcorper. Proin vitae lectus interdum, euismod sapien sed, venenatis nibh. Pellentesque justo urna, gravida vel pulvinar at, bibendum finibus urna. Class aptent taciti sociosqu ad litora torquent per conubia nostra, per inceptos himenaeos. Aenean tincidunt, velit sit amet rutrum semper, mi velit posuere massa, id porta tellus eros non tellus. Nunc blandit dui nisi, a lobortis tortor facilisis condimentum. Aliquam erat volutpat.

Vestibulum in mattis leo. Pellentesque nec nibh quis magna consectetur viverra ac et mauris. Nunc porta pretium rutrum. Vivamus hendrerit, lorem sed maximus cursus, velit enim luctus felis, a luctus ante felis non est. Sed dignissim libero orci, nec mollis libero hendrerit vel. Mauris eros lorem, efficitur placerat suscipit quis, tincidunt sed nisl. Proin sollicitudin sem sed eros pretium, in suscipit nulla fermentum. Nulla rutrum et tellus dignissim pretium. Aenean ac sagittis magna. Vestibulum convallis vestibulum diam, imperdiet iaculis dolor faucibus eu. Vivamus mattis turpis in ligula rhoncus gravida. Quisque aliquet, massa nec imperdiet eleifend, arcu felis efficitur nisl, sed malesuada lorem libero sed sem. Vivamus vitae egestas ex, ac condimentum diam. Donec porttitor lacus id metus luctus, sed sagittis augue eleifend. Quisque hendrerit maximus finibus. Fusce mollis sodales porttitor.

    \begin{figure}
        \centering
        \includegraphics[width = 4in, keepaspectratio]{Figures/stmTipShapes.pdf}
        \caption{When the tip goes wrong. }
        \label{fig:tip shapes}
    \end{figure}


Sed rhoncus orci vitae imperdiet dapibus. Ut porta nulla at purus vulputate, non lobortis lectus pretium. Fusce at augue sit amet arcu commodo fringilla. Cras urna odio, volutpat eu feugiat a, maximus tincidunt dui. Nulla semper ligula quis orci ultricies molestie. Suspendisse a nisl auctor, consectetur eros in, facilisis nunc. Praesent pulvinar placerat purus quis pulvinar. Aenean tempus pharetra elit vel viverra. Aenean sagittis, nisi vel pharetra faucibus, mauris ex venenatis quam, vel varius massa nisi sed lacus. Pellentesque habitant morbi tristique senectus et netus et malesuada fames ac turpis egestas. In euismod rhoncus turpis, in varius risus vehicula id.

Ut leo diam, tristique ut luctus et, rutrum vel risus. Praesent vel finibus erat, nec tincidunt elit. Sed auctor tristique turpis, nec lacinia lectus dignissim sed. Pellentesque tristique rutrum consequat. Sed at viverra elit. Vestibulum metus nisl, commodo non bibendum ut, tincidunt in metus. Fusce ligula sapien, iaculis sit amet consequat eget, scelerisque venenatis nulla. Vestibulum ante ipsum primis in faucibus orci luctus et ultrices posuere cubilia curae; Mauris non nisi ac diam varius hendrerit at eleifend lectus. Suspendisse leo nibh, tristique sit amet velit in, elementum molestie lectus. Nullam a turpis molestie eros sodales ultricies non ut risus. Pellentesque habitant morbi tristique senectus et netus et malesuada fames ac turpis egestas. Nunc condimentum ut leo non porttitor. Interdum et malesuada fames ac ante ipsum primis in faucibus.

Lorem ipsum dolor sit amet, consectetur adipiscing elit. Nam pulvinar rhoncus libero. Orci varius natoque penatibus et magnis dis parturient montes, nascetur ridiculus mus. Pellentesque congue, risus eu viverra scelerisque, nibh dolor porttitor tellus, eu vestibulum ligula lorem vel sem. Proin nisl nibh, blandit quis lorem eget, ultricies tempus lectus. Aliquam erat volutpat. Aenean luctus efficitur arcu id interdum. Mauris iaculis tempus ex in interdum. Phasellus a elementum tortor. Morbi facilisis urna nec ligula sagittis faucibus.

Donec facilisis, magna non fermentum auctor, tellus lacus aliquet tortor, eget pulvinar massa orci lobortis arcu. Proin finibus turpis ultricies ultrices posuere. Lorem ipsum dolor sit amet, consectetur adipiscing elit. Vestibulum tristique posuere neque. Duis scelerisque libero sit amet leo sodales, malesuada ullamcorper tellus ultricies. Cras non dapibus nisi. Integer ante augue, elementum vel interdum non, vestibulum eget magna. Vivamus laoreet, nisl eu iaculis venenatis, urna quam lobortis risus, vel ultrices velit lacus nec mauris. Curabitur volutpat ex in varius interdum. Aenean ut dignissim augue.

Nullam non tincidunt nulla, in egestas eros. Sed tincidunt, erat eget vehicula finibus, est ante commodo velit, quis mollis ipsum nulla in risus. Vivamus vestibulum lacinia tortor, non consectetur velit convallis eget. In sagittis turpis erat, tempus viverra magna feugiat eget. Phasellus et ante leo. Vestibulum viverra arcu orci, in varius lacus posuere et. Nulla facilisi. Pellentesque volutpat euismod urna sed suscipit. Donec sit amet turpis at sapien tristique egestas. Nullam rhoncus sit amet nisi quis fermentum. Integer non nunc non nisl viverra accumsan. Nullam vel sapien metus. Proin efficitur ultrices sagittis.

Fusce lobortis lectus vitae risus laoreet, quis imperdiet est gravida. Fusce et dui sit amet dolor elementum tristique. Etiam felis urna, egestas vel sapien sed, interdum dictum lacus. Nullam eu auctor augue. Aliquam mollis, ante quis mollis rhoncus, elit magna volutpat orci, eget malesuada urna sem sit amet erat. Cras porttitor eros ornare est placerat, in luctus nisi pellentesque. Pellentesque rhoncus augue vel ornare pharetra. Phasellus ac elementum massa.



%%%%%%%%%%%%%%%%%%%%%%%%%%%%%%%%%%%%%%%%%%%%%%%%%%%%%%%%%%%%%%%%%%%%%%%%%%
\section{How to Spectra}

Nunc eu nulla placerat, posuere arcu ut, elementum tellus. Donec non tortor et metus commodo vulputate rutrum porttitor odio. In sed condimentum purus. Pellentesque pulvinar tortor hendrerit pulvinar euismod. Nam vehicula diam odio, vel molestie ligula dictum et. Cras et orci diam. Etiam quis tortor orci. Donec vel urna purus.

Proin mauris tellus, molestie ut nibh ac, tristique ultrices tortor. Fusce faucibus, justo vel dapibus lobortis, velit justo venenatis augue, sed ornare lectus massa vel nulla. Integer vel tellus arcu. Maecenas congue ac nibh eu eleifend. Pellentesque habitant morbi tristique senectus et netus et malesuada fames ac turpis egestas. Maecenas imperdiet sapien nec lacus sodales faucibus. Ut lacinia est at quam tincidunt, vitae lacinia enim facilisis. Donec auctor turpis nec risus consectetur pharetra. Integer pretium ac nibh id elementum. Aenean suscipit ac tellus eu facilisis. Nunc cursus, lacus id rutrum tristique, ipsum diam maximus tortor, a hendrerit nunc neque eu augue. Orci varius natoque penatibus et magnis dis parturient montes, nascetur ridiculus mus. Suspendisse potenti. Nullam ultricies sed mi vel condimentum. Nam condimentum, lacus quis fringilla suscipit, ligula arcu placerat tortor, a congue ex eros vitae mauris.


    \begin{figure}
        \centering
        \includegraphics[width = 4in, keepaspectratio]{Figures/stmTunnelingDOS.png}
        \caption{STM tunneling graphic. }
        \label{fig:stm}
    \end{figure}

    \begin{figure}
        \centering
        \includegraphics[width = 4in, keepaspectratio]{Figures/LDOSvsTip.png}
        \caption{I'm such an artist. }
        \label{fig:stm}
    \end{figure}


Morbi molestie nunc eleifend pretium volutpat. Donec pharetra nunc ut dignissim ullamcorper. Proin vitae lectus interdum, euismod sapien sed, venenatis nibh. Pellentesque justo urna, gravida vel pulvinar at, bibendum finibus urna. Class aptent taciti sociosqu ad litora torquent per conubia nostra, per inceptos himenaeos. Aenean tincidunt, velit sit amet rutrum semper, mi velit posuere massa, id porta tellus eros non tellus. Nunc blandit dui nisi, a lobortis tortor facilisis condimentum. Aliquam erat volutpat.

Vestibulum in mattis leo. Pellentesque nec nibh quis magna consectetur viverra ac et mauris. Nunc porta pretium rutrum. Vivamus hendrerit, lorem sed maximus cursus, velit enim luctus felis, a luctus ante felis non est. Sed dignissim libero orci, nec mollis libero hendrerit vel. Mauris eros lorem, efficitur placerat suscipit quis, tincidunt sed nisl. Proin sollicitudin sem sed eros pretium, in suscipit nulla fermentum. Nulla rutrum et tellus dignissim pretium. Aenean ac sagittis magna. Vestibulum convallis vestibulum diam, imperdiet iaculis dolor faucibus eu. Vivamus mattis turpis in ligula rhoncus gravida. Quisque aliquet, massa nec imperdiet eleifend, arcu felis efficitur nisl, sed malesuada lorem libero sed sem. Vivamus vitae egestas ex, ac condimentum diam. Donec porttitor lacus id metus luctus, sed sagittis augue eleifend. Quisque hendrerit maximus finibus. Fusce mollis sodales porttitor.

    \begin{figure}
        \centering
        \includegraphics[width = 4in, keepaspectratio]{Figures/STSgraph.pdf}
        \caption{What spectra looks like}
        \label{fig:tip shapes}
    \end{figure}

Sed rhoncus orci vitae imperdiet dapibus. Ut porta nulla at purus vulputate, non lobortis lectus pretium. Fusce at augue sit amet arcu commodo fringilla. Cras urna odio, volutpat eu feugiat a, maximus tincidunt dui. Nulla semper ligula quis orci ultricies molestie. Suspendisse a nisl auctor, consectetur eros in, facilisis nunc. Praesent pulvinar placerat purus quis pulvinar. Aenean tempus pharetra elit vel viverra. Aenean sagittis, nisi vel pharetra faucibus, mauris ex venenatis quam, vel varius massa nisi sed lacus. Pellentesque habitant morbi tristique senectus et netus et malesuada fames ac turpis egestas. In euismod rhoncus turpis, in varius risus vehicula id.

Ut leo diam, tristique ut luctus et, rutrum vel risus. Praesent vel finibus erat, nec tincidunt elit. Sed auctor tristique turpis, nec lacinia lectus dignissim sed. Pellentesque tristique rutrum consequat. Sed at viverra elit. Vestibulum metus nisl, commodo non bibendum ut, tincidunt in metus. Fusce ligula sapien, iaculis sit amet consequat eget, scelerisque venenatis nulla. Vestibulum ante ipsum primis in faucibus orci luctus et ultrices posuere cubilia curae; Mauris non nisi ac diam varius hendrerit at eleifend lectus. Suspendisse leo nibh, tristique sit amet velit in, elementum molestie lectus. Nullam a turpis molestie eros sodales ultricies non ut risus. Pellentesque habitant morbi tristique senectus et netus et malesuada fames ac turpis egestas. Nunc condimentum ut leo non porttitor. Interdum et malesuada fames ac ante ipsum primis in faucibus.
%\chapter{Topology and Spectra of Defects}
\label{ch:topoandspectra}

\section{brace yourselves there's gonna be a lotta figures}

Lorem ipsum dolor sit amet, consectetur adipiscing elit. Nam pulvinar rhoncus libero. Orci varius natoque penatibus et magnis dis parturient montes, nascetur ridiculus mus. Pellentesque congue, risus eu viverra scelerisque, nibh dolor porttitor tellus, eu vestibulum ligula lorem vel sem. Proin nisl nibh, blandit quis lorem eget, ultricies tempus lectus. Aliquam erat volutpat. Aenean luctus efficitur arcu id interdum. Mauris iaculis tempus ex in interdum. Phasellus a elementum tortor. Morbi facilisis urna nec ligula sagittis faucibus.

Donec facilisis, magna non fermentum auctor, tellus lacus aliquet tortor, eget pulvinar massa orci lobortis arcu. Proin finibus turpis ultricies ultrices posuere. Lorem ipsum dolor sit amet, consectetur adipiscing elit. Vestibulum tristique posuere neque. Duis scelerisque libero sit amet leo sodales, malesuada ullamcorper tellus ultricies. Cras non dapibus nisi. Integer ante augue, elementum vel interdum non, vestibulum eget magna. Vivamus laoreet, nisl eu iaculis venenatis, urna quam lobortis risus, vel ultrices velit lacus nec mauris. Curabitur volutpat ex in varius interdum. Aenean ut dignissim augue.

Nullam non tincidunt nulla, in egestas eros. Sed tincidunt, erat eget vehicula finibus, est ante commodo velit, quis mollis ipsum nulla in risus. Vivamus vestibulum lacinia tortor, non consectetur velit convallis eget. In sagittis turpis erat, tempus viverra magna feugiat eget. Phasellus et ante leo. Vestibulum viverra arcu orci, in varius lacus posuere et. Nulla facilisi. Pellentesque volutpat euismod urna sed suscipit. Donec sit amet turpis at sapien tristique egestas. Nullam rhoncus sit amet nisi quis fermentum. Integer non nunc non nisl viverra accumsan. Nullam vel sapien metus. Proin efficitur ultrices sagittis.



%%%%%%%%%%%%%%%%%%%%%%%%%%%%%%%%%%%%%%%%%%%%%%%%%%%%%%%%%%%%%%%%%%%%%%%%
\section{Terrace Definition}

$PtSn_4$ has six distinct layers within its unit cell along the c direction. This view of the crystal lattice can be subdivided into a top and bottom that we call section $a$ and $b$ respectively. In section $a$ the first layer is of $Sn$, dubbed $Sn_1$, underneath is a square lattice of $Pt$ followed by another layer of $Sn$, dubbed $Sn_2$, that has a mirror symmetry to the first layer. Section $b$ has the same order, but it's $Sn_1$ and $Sn_2$ layers have transnational symmetry by one lattice site for section $a$'s $Sn_2$ and $Sn_1$ respectively. Section $b$'s $Pt$ layer is offset by half a unit cell respective to section $a$'s $Pt$ layer. This distinction is important as different layers with different nearest neighbors have different spectroscopies, which plays a key role in defining which atoms are on which termination.

The terrace heights measured did not align with the known height of $PtSn_4$ (reference) as seen in (fuck it insert another fig or refernce a figure). By multiplying the heights we've seen by a factor of 0.86 (ish) the correct heights can be found. STM piezos need to be calibrated, and this is most likely the correct calibration ratio. It will need to be confirmed by scanning on a well known material, such as $Au$(111)

\emph{insert figure comparing side profile and top profile of $PtSn_4$ layer by layer.}

Fusce lobortis lectus vitae risus laoreet, quis imperdiet est gravida. Fusce et dui sit amet dolor elementum tristique. Etiam felis urna, egestas vel sapien sed, interdum dictum lacus. Nullam eu auctor augue. Aliquam mollis, ante quis mollis rhoncus, elit magna volutpat orci, eget malesuada urna sem sit amet erat. Cras porttitor eros ornare est placerat, in luctus nisi pellentesque. Pellentesque rhoncus augue vel ornare pharetra. Phasellus ac elementum massa.

    \begin{figure}[h]
        \centering
        \includegraphics[width = 4in, keepaspectratio]{Figures/stepEdges.png}
        \caption{Different step edge photos}
        \label{fig:stm}
    \end{figure}
    \begin{figure}
        \centering
        \includegraphics[width = 4in, keepaspectratio]{Figures/terraceCharts.png}
        \caption{correlating heights}
        \label{fig:stm}
    \end{figure} 


Nunc eu nulla placerat, posuere arcu ut, elementum tellus. Donec non tortor et metus commodo vulputate rutrum porttitor odio. In sed condimentum purus. Pellentesque pulvinar tortor hendrerit pulvinar euismod. Nam vehicula diam odio, vel molestie ligula dictum et. Cras et orci diam. Etiam quis tortor orci. Donec vel urna purus.



%%%%%%%%%%%%%%%%%%%%%%%%%%%%%%%%%%%%%%%%%%%%%%%%%%%%%%%%%%%%%%%%%%%%%%%%%%
\section{Lattice Overlay}

Proin mauris tellus, molestie ut nibh ac, tristique ultrices tortor. Fusce faucibus, justo vel dapibus lobortis, velit justo venenatis augue, sed ornare lectus massa vel nulla. Integer vel tellus arcu. Maecenas congue ac nibh eu eleifend. Pellentesque habitant morbi tristique senectus et netus et malesuada fames ac turpis egestas. Maecenas imperdiet sapien nec lacus sodales faucibus. Ut lacinia est at quam tincidunt, vitae lacinia enim facilisis. Donec auctor turpis nec risus consectetur pharetra. Integer pretium ac nibh id elementum. Aenean suscipit ac tellus eu facilisis. Nunc cursus, lacus id rutrum tristique, ipsum diam maximus tortor, a hendrerit nunc neque eu augue. Orci varius natoque penatibus et magnis dis parturient montes, nascetur ridiculus mus. Suspendisse potenti. Nullam ultricies sed mi vel condimentum. Nam condimentum, lacus quis fringilla suscipit, ligula arcu placerat tortor, a congue ex eros vitae mauris.

    \begin{figure}
        \centering
        \includegraphics[width = 4in, keepaspectratio]{Figures/latticeViews.png}
        \caption{Different step edge photos}
        \label{fig:stm}
    \end{figure}

    \begin{figure}
        \centering
        \includegraphics[width = 4in, keepaspectratio]{Figures/latticeOverlay.png}
        \caption{lattice overlay, but of $PtSn_4$ not $LiFeAs$}
        \label{fig:stm}
    \end{figure}
    
Morbi molestie nunc eleifend pretium volutpat. Donec pharetra nunc ut dignissim ullamcorper. Proin vitae lectus interdum, euismod sapien sed, venenatis nibh. Pellentesque justo urna, gravida vel pulvinar at, bibendum finibus urna. Class aptent taciti sociosqu ad litora torquent per conubia nostra, per inceptos himenaeos. Aenean tincidunt, velit sit amet rutrum semper, mi velit posuere massa, id porta tellus eros non tellus. Nunc blandit dui nisi, a lobortis tortor facilisis condimentum. Aliquam erat volutpat.

Vestibulum in mattis leo. Pellentesque nec nibh quis magna consectetur viverra ac et mauris. Nunc porta pretium rutrum. Vivamus hendrerit, lorem sed maximus cursus, velit enim luctus felis, a luctus ante felis non est. Sed dignissim libero orci, nec mollis libero hendrerit vel. Mauris eros lorem, efficitur placerat suscipit quis, tincidunt sed nisl. Proin sollicitudin sem sed eros pretium, in suscipit nulla fermentum. Nulla rutrum et tellus dignissim pretium. Aenean ac sagittis magna. Vestibulum convallis vestibulum diam, imperdiet iaculis dolor faucibus eu. Vivamus mattis turpis in ligula rhoncus gravida. Quisque aliquet, massa nec imperdiet eleifend, arcu felis efficitur nisl, sed malesuada lorem libero sed sem. Vivamus vitae egestas ex, ac condimentum diam. Donec porttitor lacus id metus luctus, sed sagittis augue eleifend. Quisque hendrerit maximus finibus. Fusce mollis sodales porttitor.

Sed rhoncus orci vitae imperdiet dapibus. Ut porta nulla at purus vulputate, non lobortis lectus pretium. Fusce at augue sit amet arcu commodo fringilla. Cras urna odio, volutpat eu feugiat a, maximus tincidunt dui. Nulla semper ligula quis orci ultricies molestie. Suspendisse a nisl auctor, consectetur eros in, facilisis nunc. Praesent pulvinar placerat purus quis pulvinar. Aenean tempus pharetra elit vel viverra. Aenean sagittis, nisi vel pharetra faucibus, mauris ex venenatis quam, vel varius massa nisi sed lacus. Pellentesque habitant morbi tristique senectus et netus et malesuada fames ac turpis egestas. In euismod rhoncus turpis, in varius risus vehicula id.

Ut leo diam, tristique ut luctus et, rutrum vel risus. Praesent vel finibus erat, nec tincidunt elit. Sed auctor tristique turpis, nec lacinia lectus dignissim sed. Pellentesque tristique rutrum consequat. Sed at viverra elit. Vestibulum metus nisl, commodo non bibendum ut, tincidunt in metus. Fusce ligula sapien, iaculis sit amet consequat eget, scelerisque venenatis nulla. Vestibulum ante ipsum primis in faucibus orci luctus et ultrices posuere cubilia curae; Mauris non nisi ac diam varius hendrerit at eleifend lectus. Suspendisse leo nibh, tristique sit amet velit in, elementum molestie lectus. Nullam a turpis molestie eros sodales ultricies non ut risus. Pellentesque habitant morbi tristique senectus et netus et malesuada fames ac turpis egestas. Nunc condimentum ut leo non porttitor. Interdum et malesuada fames ac ante ipsum primis in faucibus.



%%%%%%%%%%%%%%%%%%%%%%%%%%%%%%%%%%%%%%%%%%%%%%%%%%%%%%%%%%%%%%%%%%%%%
\section{Defect Dive In}


Morbi molestie nunc eleifend pretium volutpat. Donec pharetra nunc ut dignissim ullamcorper. Proin vitae lectus interdum, euismod sapien sed, venenatis nibh. Pellentesque justo urna, gravida vel pulvinar at, bibendum finibus urna. Class aptent taciti sociosqu ad litora torquent per conubia nostra, per inceptos himenaeos. Aenean tincidunt, velit sit amet rutrum semper, mi velit posuere massa, id porta tellus eros non tellus. Nunc blandit dui nisi, a lobortis tortor facilisis condimentum. Aliquam erat volutpat.

\subsection{Butterfly Defect}

What we have labeled the butterfly defect topologically has a bright spot in the middle, followed by two dark lines on either side, followed with semi bright edges. These defects have a two fold symmetry and have been noticed to rotate by $90^o$.

\emph{insert a bunch of butterfly defect figures}

\subsection{Mushroom Defect}
Vestibulum in mattis leo. Pellentesque nec nibh quis magna consectetur viverra ac et mauris. Nunc porta pretium rutrum. Vivamus hendrerit, lorem sed maximus cursus, velit enim luctus felis, a luctus ante felis non est. Sed dignissim libero orci, nec mollis libero hendrerit vel. 

\emph{insert a bunch of defect topo figures}

\subsection{Spearhead Defect}
Vestibulum in mattis leo. Pellentesque nec nibh quis magna consectetur viverra ac et mauris. Nunc porta pretium rutrum. Vivamus hendrerit, lorem sed maximus cursus, velit enim luctus felis, a luctus ante felis non est. Sed dignissim libero orci, nec mollis libero hendrerit vel. 

\emph{insert a bunch of defect topo figures}

\subsection{Crescent Defect}
Vestibulum in mattis leo. Pellentesque nec nibh quis magna consectetur viverra ac et mauris. Nunc porta pretium rutrum. Vivamus hendrerit, lorem sed maximus cursus, velit enim luctus felis, a luctus ante felis non est. Sed dignissim libero orci, nec mollis libero hendrerit vel. 

\emph{insert a bunch of defect topo figures}

\subsection{Croissant Defect}
Vestibulum in mattis leo. Pellentesque nec nibh quis magna consectetur viverra ac et mauris. Nunc porta pretium rutrum. Vivamus hendrerit, lorem sed maximus cursus, velit enim luctus felis, a luctus ante felis non est. Sed dignissim libero orci, nec mollis libero hendrerit vel. 

\emph{insert a bunch of defect topo figures}

\subsection{4-Fold Symmetry Defect}
Vestibulum in mattis leo. Pellentesque nec nibh quis magna consectetur viverra ac et mauris. Nunc porta pretium rutrum. Vivamus hendrerit, lorem sed maximus cursus, velit enim luctus felis, a luctus ante felis non est. Sed dignissim libero orci, nec mollis libero hendrerit vel. 

\emph{insert a bunch of defect topo figures}

    \begin{figure}
        \centering
        \includegraphics[width = 5in, keepaspectratio]{Figures/topoplusSTSmultiGraph.pdf}
        \caption{A bunch of these for each type of defect, multiple STS showings}
        \label{fig:stm}
    \end{figure}
    


Vestibulum in mattis leo. Pellentesque nec nibh quis magna consectetur viverra ac et mauris. Nunc porta pretium rutrum. Vivamus hendrerit, lorem sed maximus cursus, velit enim luctus felis, a luctus ante felis non est. Sed dignissim libero orci, nec mollis libero hendrerit vel. Mauris eros lorem, efficitur placerat suscipit quis, tincidunt sed nisl. Proin sollicitudin sem sed eros pretium, in suscipit nulla fermentum. Nulla rutrum et tellus dignissim pretium. Aenean ac sagittis magna. Vestibulum convallis vestibulum diam, imperdiet iaculis dolor faucibus eu. Vivamus mattis turpis in ligula rhoncus gravida. Quisque aliquet, massa nec imperdiet eleifend, arcu felis efficitur nisl, sed malesuada lorem libero sed sem. Vivamus vitae egestas ex, ac condimentum diam. Donec porttitor lacus id metus luctus, sed sagittis augue eleifend. Quisque hendrerit maximus finibus. Fusce mollis sodales porttitor.

Sed rhoncus orci vitae imperdiet dapibus. Ut porta nulla at purus vulputate, non lobortis lectus pretium. Fusce at augue sit amet arcu commodo fringilla. Cras urna odio, volutpat eu feugiat a, maximus tincidunt dui. Nulla semper ligula quis orci ultricies molestie. Suspendisse a nisl auctor, consectetur eros in, facilisis nunc. Praesent pulvinar placerat purus quis pulvinar. Aenean tempus pharetra elit vel viverra. Aenean sagittis, nisi vel pharetra faucibus, mauris ex venenatis quam, vel varius massa nisi sed lacus. Pellentesque habitant morbi tristique senectus et netus et malesuada fames ac turpis egestas. In euismod rhoncus turpis, in varius risus vehicula id.

.
%\chapter{Analysis on Rotating Defects}
\label{ch:rotdefect}


\section{Bish guess what}

Lorem ipsum dolor sit amet, consectetur adipiscing elit. Nam pulvinar rhoncus libero. Orci varius natoque penatibus et magnis dis parturient montes, nascetur ridiculus mus. Pellentesque congue, risus eu viverra scelerisque, nibh dolor porttitor tellus, eu vestibulum ligula lorem vel sem. Proin nisl nibh, blandit quis lorem eget, ultricies tempus lectus. Aliquam erat volutpat. Aenean luctus efficitur arcu id interdum. Mauris iaculis tempus ex in interdum. Phasellus a elementum tortor. Morbi facilisis urna nec ligula sagittis faucibus.

Donec facilisis, magna non fermentum auctor, tellus lacus aliquet tortor, eget pulvinar massa orci lobortis arcu. Proin finibus turpis ultricies ultrices posuere. Lorem ipsum dolor sit amet, consectetur adipiscing elit. Vestibulum tristique posuere neque. Duis scelerisque libero sit amet leo sodales, malesuada ullamcorper tellus ultricies. Cras non dapibus nisi. Integer ante augue, elementum vel interdum non, vestibulum eget magna. Vivamus laoreet, nisl eu iaculis venenatis, urna quam lobortis risus, vel ultrices velit lacus nec mauris. Curabitur volutpat ex in varius interdum. Aenean ut dignissim augue.

Nullam non tincidunt nulla, in egestas eros. Sed tincidunt, erat eget vehicula finibus, est ante commodo velit, quis mollis ipsum nulla in risus. Vivamus vestibulum lacinia tortor, non consectetur velit convallis eget. In sagittis turpis erat, tempus viverra magna feugiat eget. Phasellus et ante leo. Vestibulum viverra arcu orci, in varius lacus posuere et. Nulla facilisi. Pellentesque volutpat euismod urna sed suscipit. Donec sit amet turpis at sapien tristique egestas. Nullam rhoncus sit amet nisi quis fermentum. Integer non nunc non nisl viverra accumsan. Nullam vel sapien metus. Proin efficitur ultrices sagittis.

    \begin{figure}
        \centering
        \includegraphics[width = 4.5in, keepaspectratio]{Figures/firstRotation.png}
        \caption{When shit went down}
        \label{fig:stm}
    \end{figure}

Fusce lobortis lectus vitae risus laoreet, quis imperdiet est gravida. Fusce et dui sit amet dolor elementum tristique. Etiam felis urna, egestas vel sapien sed, interdum dictum lacus. Nullam eu auctor augue. Aliquam mollis, ante quis mollis rhoncus, elit magna volutpat orci, eget malesuada urna sem sit amet erat. Cras porttitor eros ornare est placerat, in luctus nisi pellentesque. Pellentesque rhoncus augue vel ornare pharetra. Phasellus ac elementum massa.

Nunc eu nulla placerat, posuere arcu ut, elementum tellus. Donec non tortor et metus commodo vulputate rutrum porttitor odio. In sed condimentum purus. Pellentesque pulvinar tortor hendrerit pulvinar euismod. Nam vehicula diam odio, vel molestie ligula dictum et. Cras et orci diam. Etiam quis tortor orci. Donec vel urna purus.

\section{E Field or State Dependent thing?}

Proin mauris tellus, molestie ut nibh ac, tristique ultrices tortor. Fusce faucibus, justo vel dapibus lobortis, velit justo venenatis augue, sed ornare lectus massa vel nulla. Integer vel tellus arcu. Maecenas congue ac nibh eu eleifend. Pellentesque habitant morbi tristique senectus et netus et malesuada fames ac turpis egestas. Maecenas imperdiet sapien nec lacus sodales faucibus. Ut lacinia est at quam tincidunt, vitae lacinia enim facilisis. Donec auctor turpis nec risus consectetur pharetra. Integer pretium ac nibh id elementum. Aenean suscipit ac tellus eu facilisis. Nunc cursus, lacus id rutrum tristique, ipsum diam maximus tortor, a hendrerit nunc neque eu augue. Orci varius natoque penatibus et magnis dis parturient montes, nascetur ridiculus mus. Suspendisse potenti. Nullam ultricies sed mi vel condimentum. Nam condimentum, lacus quis fringilla suscipit, ligula arcu placerat tortor, a congue ex eros vitae mauris.

Morbi molestie nunc eleifend pretium volutpat. Donec pharetra nunc ut dignissim ullamcorper. Proin vitae lectus interdum, euismod sapien sed, venenatis nibh. Pellentesque justo urna, gravida vel pulvinar at, bibendum finibus urna. Class aptent taciti sociosqu ad litora torquent per conubia nostra, per inceptos himenaeos. Aenean tincidunt, velit sit amet rutrum semper, mi velit posuere massa, id porta tellus eros non tellus. Nunc blandit dui nisi, a lobortis tortor facilisis condimentum. Aliquam erat volutpat.

    \begin{figure}
        \centering
        \includegraphics[width = 1in, keepaspectratio]{Figures/electronwideeyed.png}
        \caption{holy shit what's happening (Show stats of croissant flippage)}
        \label{fig:stm}
    \end{figure}

Vestibulum in mattis leo. Pellentesque nec nibh quis magna consectetur viverra ac et mauris. Nunc porta pretium rutrum. Vivamus hendrerit, lorem sed maximus cursus, velit enim luctus felis, a luctus ante felis non est. Sed dignissim libero orci, nec mollis libero hendrerit vel. Mauris eros lorem, efficitur placerat suscipit quis, tincidunt sed nisl. Proin sollicitudin sem sed eros pretium, in suscipit nulla fermentum. Nulla rutrum et tellus dignissim pretium. Aenean ac sagittis magna. Vestibulum convallis vestibulum diam, imperdiet iaculis dolor faucibus eu. Vivamus mattis turpis in ligula rhoncus gravida. Quisque aliquet, massa nec imperdiet eleifend, arcu felis efficitur nisl, sed malesuada lorem libero sed sem. Vivamus vitae egestas ex, ac condimentum diam. Donec porttitor lacus id metus luctus, sed sagittis augue eleifend. Quisque hendrerit maximus finibus. Fusce mollis sodales porttitor.

Sed rhoncus orci vitae imperdiet dapibus. Ut porta nulla at purus vulputate, non lobortis lectus pretium. Fusce at augue sit amet arcu commodo fringilla. Cras urna odio, volutpat eu feugiat a, maximus tincidunt dui. Nulla semper ligula quis orci ultricies molestie. Suspendisse a nisl auctor, consectetur eros in, facilisis nunc. Praesent pulvinar placerat purus quis pulvinar. Aenean tempus pharetra elit vel viverra. Aenean sagittis, nisi vel pharetra faucibus, mauris ex venenatis quam, vel varius massa nisi sed lacus. Pellentesque habitant morbi tristique senectus et netus et malesuada fames ac turpis egestas. In euismod rhoncus turpis, in varius risus vehicula id.

Ut leo diam, tristique ut luctus et, rutrum vel risus. Praesent vel finibus erat, nec tincidunt elit. Sed auctor tristique turpis, nec lacinia lectus dignissim sed. Pellentesque tristique rutrum consequat. Sed at viverra elit. Vestibulum metus nisl, commodo non bibendum ut, tincidunt in metus. Fusce ligula sapien, iaculis sit amet consequat eget, scelerisque venenatis nulla. Vestibulum ante ipsum primis in faucibus orci luctus et ultrices posuere cubilia curae; Mauris non nisi ac diam varius hendrerit at eleifend lectus. Suspendisse leo nibh, tristique sit amet velit in, elementum molestie lectus. Nullam a turpis molestie eros sodales ultricies non ut risus. Pellentesque habitant morbi tristique senectus et netus et malesuada fames ac turpis egestas. Nunc condimentum ut leo non porttitor. Interdum et malesuada fames ac ante ipsum primis in faucibus.Vestibulum in mattis leo. Pellentesque nec nibh quis magna consectetur viverra ac et mauris. Nunc porta pretium rutrum. Vivamus hendrerit, lorem sed maximus cursus, velit enim luctus felis, a luctus ante felis non est. Sed dignissim libero orci, nec mollis libero hendrerit vel. Mauris eros lorem, efficitur placerat suscipit quis, tincidunt sed nisl. Proin sollicitudin sem sed eros pretium, in suscipit nulla fermentum. Nulla rutrum et tellus dignissim pretium. Aenean ac sagittis magna. Vestibulum convallis vestibulum diam, imperdiet iaculis dolor faucibus eu. Vivamus mattis turpis in ligula rhoncus gravida. Quisque aliquet, massa nec imperdiet eleifend, arcu felis efficitur nisl, sed malesuada lorem libero sed sem. Vivamus vitae egestas ex, ac condimentum diam. Donec porttitor lacus id metus luctus, sed sagittis augue eleifend. Quisque hendrerit maximus finibus. Fusce mollis sodales porttitor.

Sed rhoncus orci vitae imperdiet dapibus. Ut porta nulla at purus vulputate, non lobortis lectus pretium. Fusce at augue sit amet arcu commodo fringilla. Cras urna odio, volutpat eu feugiat a, maximus tincidunt dui. Nulla semper ligula quis orci ultricies molestie. Suspendisse a nisl auctor, consectetur eros in, facilisis nunc. Praesent pulvinar placerat purus quis pulvinar. Aenean tempus pharetra elit vel viverra. Aenean sagittis, nisi vel pharetra faucibus, mauris ex venenatis quam, vel varius massa nisi sed lacus. Pellentesque habitant morbi tristique senectus et netus et malesuada fames ac turpis egestas. In euismod rhoncus turpis, in varius risus vehicula id.

Ut leo diam, tristique ut luctus et, rutrum vel risus. Praesent vel finibus erat, nec tincidunt elit. Sed auctor tristique turpis, nec lacinia lectus dignissim sed. Pellentesque tristique rutrum consequat. Sed at viverra elit. Vestibulum metus nisl, commodo non bibendum ut, tincidunt in metus. Fusce ligula sapien, iaculis sit amet consequat eget, scelerisque venenatis nulla. Vestibulum ante ipsum primis in faucibus orci luctus et ultrices posuere cubilia curae; Mauris non nisi ac diam varius hendrerit at eleifend lectus. Suspendisse leo nibh, tristique sit amet velit in, elementum molestie lectus. Nullam a turpis molestie eros sodales ultricies non ut risus. Pellentesque habitant morbi tristique senectus et netus et malesuada fames ac turpis egestas. Nunc condimentum ut leo non porttitor. Interdum et malesuada fames ac ante ipsum primis in faucibus.Vestibulum in mattis leo. Pellentesque nec nibh quis magna consectetur viverra ac et mauris. Nunc porta pretium rutrum. Vivamus hendrerit, lorem sed maximus cursus, velit enim luctus felis, a luctus ante felis non est. Sed dignissim libero orci, nec mollis libero hendrerit vel. Mauris eros lorem, efficitur placerat suscipit quis, tincidunt sed nisl. Proin sollicitudin sem sed eros pretium, in suscipit nulla fermentum. Nulla rutrum et tellus dignissim pretium. Aenean ac sagittis magna. Vestibulum convallis vestibulum diam, imperdiet iaculis dolor faucibus eu. Vivamus mattis turpis in ligula rhoncus gravida. Quisque aliquet, massa nec imperdiet eleifend, arcu felis efficitur nisl, sed malesuada lorem libero sed sem. Vivamus vitae egestas ex, ac condimentum diam. Donec porttitor lacus id metus luctus, sed sagittis augue eleifend. Quisque hendrerit maximus finibus. Fusce mollis sodales porttitor.

    \begin{figure}[h]
        \centering
        \includegraphics[width = 4in, keepaspectratio]{Figures/image.png}
        \caption{the original}
        \label{fig:stm}
    \end{figure}
    \begin{figure}[h]
        \centering
        \includegraphics[width = 4in, keepaspectratio]{Figures/image (1).png}
        \caption{the turn}
        \label{fig:stm}
    \end{figure}
    \begin{figure}[h]
        \centering
        \includegraphics[width = 4in, keepaspectratio]{Figures/image (2).png}
        \caption{The "I went and joined the witness protection program with my buddy" (maybe I'll put these all in one figure)}
        \label{fig:stm}
    \end{figure}

Sed rhoncus orci vitae imperdiet dapibus. Ut porta nulla at purus vulputate, non lobortis lectus pretium. Fusce at augue sit amet arcu commodo fringilla. Cras urna odio, volutpat eu feugiat a, maximus tincidunt dui. Nulla semper ligula quis orci ultricies molestie. Suspendisse a nisl auctor, consectetur eros in, facilisis nunc. Praesent pulvinar placerat purus quis pulvinar. Aenean tempus pharetra elit vel viverra. Aenean sagittis, nisi vel pharetra faucibus, mauris ex venenatis quam, vel varius massa nisi sed lacus. Pellentesque habitant morbi tristique senectus et netus et malesuada fames ac turpis egestas. In euismod rhoncus turpis, in varius risus vehicula id.

\section{Okay but why? what is it?}

Ut leo diam, tristique ut luctus et, rutrum vel risus. Praesent vel finibus erat, nec tincidunt elit. Sed auctor tristique turpis, nec lacinia lectus dignissim sed. Pellentesque tristique rutrum consequat. Sed at viverra elit. Vestibulum metus nisl, commodo non bibendum ut, tincidunt in metus. Fusce ligula sapien, iaculis sit amet consequat eget, scelerisque venenatis nulla. Vestibulum ante ipsum primis in faucibus orci luctus et ultrices posuere cubilia curae; Mauris non nisi ac diam varius hendrerit at eleifend lectus. Suspendisse leo nibh, tristique sit amet velit in, elementum molestie lectus. Nullam a turpis molestie eros sodales ultricies non ut risus. Pellentesque habitant morbi tristique senectus et netus et malesuada fames ac turpis egestas. Nunc condimentum ut leo non porttitor. Interdum et malesuada fames ac ante ipsum primis in faucibus.Vestibulum in mattis leo. Pellentesque nec nibh quis magna consectetur viverra ac et mauris. Nunc porta pretium rutrum. Vivamus hendrerit, lorem sed maximus cursus, velit enim luctus felis, a luctus ante felis non est. Sed dignissim libero orci, nec mollis libero hendrerit vel. Mauris eros lorem, efficitur placerat suscipit quis, tincidunt sed nisl. Proin sollicitudin sem sed eros pretium, in suscipit nulla fermentum. Nulla rutrum et tellus dignissim pretium. Aenean ac sagittis magna. Vestibulum convallis vestibulum diam, imperdiet iaculis dolor faucibus eu. Vivamus mattis turpis in ligula rhoncus gravida. Quisque aliquet, massa nec imperdiet eleifend, arcu felis efficitur nisl, sed malesuada lorem libero sed sem. Vivamus vitae egestas ex, ac condimentum diam. Donec porttitor lacus id metus luctus, sed sagittis augue eleifend. Quisque hendrerit maximus finibus. Fusce mollis sodales porttitor.

    \begin{figure}
        \centering
        \includegraphics[width = 4in, keepaspectratio]{Figures/vacancyCroissant.png}
        \caption{croissant theory}
        \label{fig:stm}
    \end{figure}

Sed rhoncus orci vitae imperdiet dapibus. Ut porta nulla at purus vulputate, non lobortis lectus pretium. Fusce at augue sit amet arcu commodo fringilla. Cras urna odio, volutpat eu feugiat a, maximus tincidunt dui. Nulla semper ligula quis orci ultricies molestie. Suspendisse a nisl auctor, consectetur eros in, facilisis nunc. Praesent pulvinar placerat purus quis pulvinar. Aenean tempus pharetra elit vel viverra. Aenean sagittis, nisi vel pharetra faucibus, mauris ex venenatis quam, vel varius massa nisi sed lacus. Pellentesque habitant morbi tristique senectus et netus et malesuada fames ac turpis egestas. In euismod rhoncus turpis, in varius risus vehicula id.

Ut leo diam, tristique ut luctus et, rutrum vel risus. Praesent vel finibus erat, nec tincidunt elit. Sed auctor tristique turpis, nec lacinia lectus dignissim sed. Pellentesque tristique rutrum consequat. Sed at viverra elit. Vestibulum metus nisl, commodo non bibendum ut, tincidunt in metus. Fusce ligula sapien, iaculis sit amet consequat eget, scelerisque venenatis nulla. Vestibulum ante ipsum primis in faucibus orci luctus et ultrices posuere cubilia curae; Mauris non nisi ac diam varius hendrerit at eleifend lectus. Suspendisse leo nibh, tristique sit amet velit in, elementum molestie lectus. Nullam a turpis molestie eros sodales ultricies non ut risus. Pellentesque habitant morbi tristique senectus et netus et malesuada fames ac turpis egestas. Nunc condimentum ut leo non porttitor. Interdum et malesuada fames ac ante ipsum primis in faucibus.
%\chapter{Conclusion}
\label{ch:conclusion}

And thus...science was done.

The $PtSn_4$ sample has been recleaved and inserted back into the STM along with a new $W$ tip. We hope to discern if the rotating defects arise in the same density as well as any additional defects that may arise. With this knowledge of defects and terminations we hope to perform detailed grid maps over more defects with a fresh tip.

We will be performing z callibration because I have no idea what the frickity frackin fuck I'm looking at and I'm V ANNOYED.

Lorem ipsum dolor sit amet, consectetur adipiscing elit. Nam pulvinar rhoncus libero. Orci varius natoque penatibus et magnis dis parturient montes, nascetur ridiculus mus. Pellentesque congue, risus eu viverra scelerisque, nibh dolor porttitor tellus, eu vestibulum ligula lorem vel sem. Proin nisl nibh, blandit quis lorem eget, ultricies tempus lectus. Aliquam erat volutpat. Aenean luctus efficitur arcu id interdum. Mauris iaculis tempus ex in interdum. Phasellus a elementum tortor. Morbi facilisis urna nec ligula sagittis faucibus.

% Generally recommended to put each chapter into a separate file
%\include{relatedwork}
%\include{model}
%\include{impl}
%\include{discussion}
%\include{conclusions}

%    3. Notes
%    4. Footnotes

%    5. Bibliography
\begin{singlespace}
\printbibliography[heading=bibintoc]
\end{singlespace}

\appendix
%    6. Appendices (including copies of all required UBC Research
%       Ethics Board's Certificates of Approval)
%\include{reb-coa}	% pdfpages is useful here
% \chapter{Supporting Materials}

This would be any supporting material not central to the dissertation.
For example:
\begin{itemize}
\item additional details of methodology and/or data;
\item diagrams of specialized equipment developed.;
\item copies of questionnaires and survey instruments.
\end{itemize}


\backmatter
%    7. Index
% See the makeindex package: the following page provides a quick overview
% <http://www.image.ufl.edu/help/latex/latex_indexes.shtml>


\end{document}
